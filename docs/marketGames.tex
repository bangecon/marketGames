% Options for packages loaded elsewhere
\PassOptionsToPackage{unicode}{hyperref}
\PassOptionsToPackage{hyphens}{url}
%
\documentclass[
]{article}
\title{An Open-Source Application to Computerize Simple Market
Efficiency Games}
\author{James T. Bang}
\date{1/28/2022}

\usepackage{amsmath,amssymb}
\usepackage{lmodern}
\usepackage{iftex}
\ifPDFTeX
  \usepackage[T1]{fontenc}
  \usepackage[utf8]{inputenc}
  \usepackage{textcomp} % provide euro and other symbols
\else % if luatex or xetex
  \usepackage{unicode-math}
  \defaultfontfeatures{Scale=MatchLowercase}
  \defaultfontfeatures[\rmfamily]{Ligatures=TeX,Scale=1}
\fi
% Use upquote if available, for straight quotes in verbatim environments
\IfFileExists{upquote.sty}{\usepackage{upquote}}{}
\IfFileExists{microtype.sty}{% use microtype if available
  \usepackage[]{microtype}
  \UseMicrotypeSet[protrusion]{basicmath} % disable protrusion for tt fonts
}{}
\makeatletter
\@ifundefined{KOMAClassName}{% if non-KOMA class
  \IfFileExists{parskip.sty}{%
    \usepackage{parskip}
  }{% else
    \setlength{\parindent}{0pt}
    \setlength{\parskip}{6pt plus 2pt minus 1pt}}
}{% if KOMA class
  \KOMAoptions{parskip=half}}
\makeatother
\usepackage{xcolor}
\IfFileExists{xurl.sty}{\usepackage{xurl}}{} % add URL line breaks if available
\IfFileExists{bookmark.sty}{\usepackage{bookmark}}{\usepackage{hyperref}}
\hypersetup{
  pdftitle={An Open-Source Application to Computerize Simple Market Efficiency Games},
  pdfauthor={James T. Bang},
  hidelinks,
  pdfcreator={LaTeX via pandoc}}
\urlstyle{same} % disable monospaced font for URLs
\usepackage[margin=1in]{geometry}
\usepackage{graphicx}
\makeatletter
\def\maxwidth{\ifdim\Gin@nat@width>\linewidth\linewidth\else\Gin@nat@width\fi}
\def\maxheight{\ifdim\Gin@nat@height>\textheight\textheight\else\Gin@nat@height\fi}
\makeatother
% Scale images if necessary, so that they will not overflow the page
% margins by default, and it is still possible to overwrite the defaults
% using explicit options in \includegraphics[width, height, ...]{}
\setkeys{Gin}{width=\maxwidth,height=\maxheight,keepaspectratio}
% Set default figure placement to htbp
\makeatletter
\def\fps@figure{htbp}
\makeatother
\setlength{\emergencystretch}{3em} % prevent overfull lines
\providecommand{\tightlist}{%
  \setlength{\itemsep}{0pt}\setlength{\parskip}{0pt}}
\setcounter{secnumdepth}{-\maxdimen} % remove section numbering
\newlength{\cslhangindent}
\setlength{\cslhangindent}{1.5em}
\newlength{\csllabelwidth}
\setlength{\csllabelwidth}{3em}
\newlength{\cslentryspacingunit} % times entry-spacing
\setlength{\cslentryspacingunit}{\parskip}
\newenvironment{CSLReferences}[2] % #1 hanging-ident, #2 entry spacing
 {% don't indent paragraphs
  \setlength{\parindent}{0pt}
  % turn on hanging indent if param 1 is 1
  \ifodd #1
  \let\oldpar\par
  \def\par{\hangindent=\cslhangindent\oldpar}
  \fi
  % set entry spacing
  \setlength{\parskip}{#2\cslentryspacingunit}
 }%
 {}
\usepackage{calc}
\newcommand{\CSLBlock}[1]{#1\hfill\break}
\newcommand{\CSLLeftMargin}[1]{\parbox[t]{\csllabelwidth}{#1}}
\newcommand{\CSLRightInline}[1]{\parbox[t]{\linewidth - \csllabelwidth}{#1}\break}
\newcommand{\CSLIndent}[1]{\hspace{\cslhangindent}#1}
\ifLuaTeX
  \usepackage{selnolig}  % disable illegal ligatures
\fi

\begin{document}
\maketitle
\begin{abstract}
In-class experiments provide instructors a powerful tool for helping
students learn and understand market principles in economics. Despite
the effectiveness of experiments, economics instructors remain slow to
adopt them in their pedagogy. One reason for this lag could be the
time-consuming process of collecting, tabulating, and presenting the
outcomes of the experiments. This paper introduces functions and
ShinyApps in R for fast, free, in-class tabulation of the results of
five in-class market simulation experiments for teaching economics.

Keywords: Experiments; technology; teaching
\end{abstract}

\hypertarget{introduction}{%
\section{Introduction}\label{introduction}}

The use of experiments as demonstrations of economic theory date back at
least as far as those conducted by Chamberlin (1948) and Smith (1962).
These studies were designed primarily to collect evidence supporting or
refuting economic models of rational behavior in market settings. Holt
(1993) summarizes this literature. While experiments remain an important
method for observing behavior to test economic hypotheses, these
experiments have also found their way into pedagogy (DeYoung 1993).

Classroom experiments offer students and instructors a fun departure
from the usual ``chalk and talk'' of explaining economic models. In
addition to entertainment value, studies have shown experiments to
increase student learning in post-test assessments (Emerson and Taylor
2004; Dickie 2006). I should note, however, that not all studies
conclude that all types of gamification improves learning by a
significant margin: Gremmen and Potters (1997) find a positive effect of
games on average, but the effect is not statistically significant, while
Dickie (2006) finds that games do significantly improve learning, but
that attaching grade incentives to the games do not contribute any
additional benefit. Moreover, Stodder (1998) expresses concern that
classroom games that penalize cooperation may teach and reinforce
unethical decision making.

Despite the potential learning and entertainment value of classroom
experiments, they remain relatively rare among the pedagogies economics
professors adopt in their classrooms (Watts and Becker 2008). Two
factors may drive some of the hesitancy among economics instructors to
implement classroom experiments. On the one hand, free resources, such
as those described in the survey of non-computerized games by Brauer and
Delemeester (2001), require significant time investments to tabulate and
summarize the results. On the other hand, automated resources,
especially those distributed by textbook publishers, impose a financial
cost on students or their institutions that instructors feel rightly
averse to asking budget-constrained students or departments to foot the
bill for.

This contribution aims to overcome these barriers by automating some of
the existing non-computerized experiments and introducing simplified
versions of classic duopoly games with the hope of increasing the
diversity of methods used by economics instructors in the classroom.
These examples only require students to be able to access a Google Form
via their browser on their computer or mobile device, which, given the
ubiquity of mobile phones among students (sometimes as their only
personal computing device), sets a fairly reasonable bar for
accessibility.

On the instructor's end, I have created a free, downloadable package
called \texttt{econGame} for the \emph{R} open-source statistical
computing program. The tabulation programs run as either stand-alone
functions in the \emph{R} console, or for demonstration purposes as a
html \emph{Shiny} app that can open in a browser tab if desired. This
allows the instructor to present the results of the experiment almost
instantaneously after the students have submitted their responses.

\hypertarget{description-of-the-experiments}{%
\section{Description of the
Experiments}\label{description-of-the-experiments}}

The models I will describe in this paper encompass two well-established
market equilibrium games known to the economics education literature,
namely simplified versions of the pit market trading game introduced by
Holt (1996) and the free entry and exit game introduced by Garratt
(2000); and three games simulating different oligopoly models (Bertrand,
Cournot, and Stackelberg). In the examples the ``payoffs'' students
receive can be awarded to the students at the end of the games as
``extra credit'' points, or instructors may choose to encourage students
to play the games strategically, but only ``for the love of the game.''
I briefly describe the delivery of the games below.

\hypertarget{pit-market-trading}{%
\subsection{Pit Market Trading}\label{pit-market-trading}}

The instructor informs the students that they own a single unit of a
eCoin currency that each of them values differently. Students then
receive a random value from 1 to 10 using a link to a Google Sheet. This
number represents their (constant) value they attach to the unit they
presently own and for if they were to acquire one more unit of
eCoin.\footnote{A template can be found at:
  \url{https://docs.google.com/spreadsheets/d/1lCmC692ajsQZoatWtgZh5QKaJ9y3pOMt15JwRFaHanU/edit\#gid=258904023}.}

Students submit their name, their value draw, a ``bid'' corresponding to
the highest amount they would pay for a second eCoin, and an ``ask''
corresponding to the lowest amount they would accept to part with the
eCoin they already own.\footnote{A template can be found at:
  \url{https://docs.google.com/forms/d/1S_F9UJ6GXttxPqDLtk8Hg0ZgzDaHMxBmc1qH3W2gKZo/edit}.
  It is important that when users create their own copies of the
  template that they use the same names in the question prompts. In case
  an instructor wants to add additional instructions or framing around
  the questions, they should add text fields.}

Students keep their consumer and producer surpluses from each round as
``extra credit'' points. \texttt{equilibriumGame} tabulates the supply
and demand schedules; calculates the equilibrium; graphs the
equilibrium; and tabulates the scores for each student.\footnote{The
  solution the piecewise constant supply and demand equilibrium uses the
  help of a C++ helper function provided by ``David'' on Stack Overflow,
  \url{https://stackoverflow.com/questions/23830906/intersection-of-two-step-functions}.}

\hypertarget{free-entry-and-exit}{%
\subsection{Free Entry and Exit}\label{free-entry-and-exit}}

The instructor informs the students that they will choose to plant corn,
soybeans, or nothing. Producing corn incurs a cost of four points, while
producing soybeans incurs a cost of 10 points. Selling a unit of corn
brings revenue equal to \[P_c = (N/2) + 6 - Q_c,\] where \(N\) equals
the number of students participating and \(Q_c\) equals the number of
students choosing to produce corn. Selling a unit of soybeans brings
revenue equal to \[P_s = (N/2) + 10 - Q_s.\] These parameters allow for
there to be a ``normal profit'' of about one point per student in each
market in equilibrium, to compensate for the risk of venturing into
self-employment and lessen the chances that students might ``win''
negative points. Students choosing to produce nothing sell their labor
in the labor market and break even.\footnote{A template can be found at:
  \url{https://docs.google.com/forms/d/1oUsLulfD5bqT6_9VVYIzLWuuQ-L4vwmC4jI-1jabOVQ/edit}.}

Students may play as many rounds as the instructor decides to continue
the game, or until the markets reach the long run equilibrium of zero
\emph{economic} profit. Usually the markets converge to the long run
equilibrium (or students begin to overthink the excercise) by the end of
the fourth or fifth round.

\hypertarget{oligoply-models}{%
\subsection{Oligoply Models}\label{oligoply-models}}

I also constructed a set of games to demonstrate and compare equilibria
in different (two-firm) oligopoly models. In each of the examples,
students work in pairs. The instructor informs the students that the
market price depends on both the strategy they choose for their ``firm''
and also the strategy their partner chooses. Each of the three examples
uses the following linear inverse demand function (the parameters of
which individual instructors may change in the options):
\[P = a + b(Q_1 + Q_2),\] where the default values for the parameters
are \(a = 10\) and \(b = -1\). Likewise, firms face the the same cost
function: \[TC = f + cQ_i,\] where \(f\) represents the fixed cost (0 by
default) and \(c\) represents the (constant) marginal cost of each
additional unit (6 by default).

\hypertarget{bertrand-duoploy}{%
\subsubsection{Bertrand Duoploy}\label{bertrand-duoploy}}

In the Bertrand game, students choose their price, and the firm with the
lowest price wins the whole market demand. If both firms choose the same
price, they split the market. Students submit their name and price
strategy through a Google Form.\footnote{A template can be found at:
  \url{https://docs.google.com/forms/d/1AykOoY6mVj17D_5CW7-BLhSgOJdGEhyfYHKHROnvdcg/edit}.}

\hypertarget{cournot-duopoly}{%
\subsubsection{Cournot Duopoly}\label{cournot-duopoly}}

In the Cournot game, students choose either to produce either a low or
high quantity by choosing the strategy to ``collude'' (low quantity) or
``defect'' (high quantity). The function that tabulates the results
assigns half of the monopolist's profit-maximizing quantity to students
who choose ``collude,'' and automatically assigns the quantity
corresponding to the best response function for students who choose
``defect'' (which depends on the output choice of their rival). Students
only need to make the simple binary choice.\footnote{A template can be
  found at:
  \url{https://docs.google.com/forms/d/1dp-tUv5rNhRpm9UjFCy_pgsD4rJJnja-QJnWnMu81DI/edit}.}

Instructors using this example for upper-level classes may (or may not)
want to edit the game settings to require students to derive the best
response functions and determine their quantity strategies themselves.

\hypertarget{stackelberg-duopoly}{%
\subsubsection{Stackelberg Duopoly}\label{stackelberg-duopoly}}

Similar to the Cournot game, students in the Stackelberg game choose to
``collude'' or ``defect.'' In contrast to the Cournot game, the
Stackelberg game reveals the leaders' strategy choice before the
follower chooses their strategy. The function again automatically
calculates the quantities corresponding to the set of binary strategy
choices to determine the payoff outcomes.{[}\^{}A template can be found
at:
\url{https://docs.google.com/forms/d/1vERPMPt_kW96JPAY6mEtkQMu6FLCgPuqoFL8i8bulYk/edit}.{]}

\hypertarget{the-econgame-package}{%
\section{The econGame Package}\label{the-econgame-package}}

The \texttt{econGame} package pulls activity data from Google the Google
Sheet file generated for the form responses. It then tabulates, scores,
and graphs the responses; and writes the grade data to a separate Google
Sheet. Users can download the package from
\url{https://github.com/bangecon/econGame}. To install the package, use
the \texttt{remotes} package and the \texttt{install\_github()}
function:

\texttt{remotes::install\_github("bangecon/econGame")}

\hypertarget{exporting-the-results}{%
\subsection{Exporting the Results}\label{exporting-the-results}}

Once the students have completed each round of a game, the instructor
should pause to tabulate the results for that round. This is especially
the case in the entry and exit game, where convergence to the long-run
equilibrium requires students to know which sector earned economic
profits in the previous round. To do this, all the instructor needs to
do is click the Google Sheets icon in the responses tab of the edit page
of the form. Figure 1 demonstrates where to find these tools in your
Google Form.

\begin{figure}
\centering
\includegraphics{Figure1.png}
\caption{Exporting Results to Google Sheets}
\end{figure}

This creates a Google Sheet that contains all of the data for tabulating
the results. In order to link to your results, you will need to copy to
your clipboard the sheet ID from the URL. Figure 2 shows where you can
find the sheet ID for your results.

\begin{figure}
\centering
\includegraphics{Figure2.png}
\caption{Finding the Sheet ID}
\end{figure}

\hypertarget{tabulating-the-results}{%
\subsection{Tabulating the Results}\label{tabulating-the-results}}

Each game has its own function to tabulate its results, and for each
function there are slightly different options that the instructor can
adjust based on any changes they might have made to the default
parameters for the exercise. The only argument the instructor needs to
provide (and that does not have a preset default) is the ID of the
Google Sheet. The syntax and outputs for the different functions are:

\begin{itemize}
\item
  Equilibrium game: \texttt{equilibriumGame(sheet\ =\ NULL,\ ...)}

  \begin{itemize}
  \item
    Arguments:

    \begin{itemize}
    \tightlist
    \item
      \texttt{sheet} (required) is a character string url corresponding
      to the Google Sheets location containing the individual
      submissions.
    \end{itemize}
  \item
    Outputs:

    \begin{itemize}
    \tightlist
    \item
      \texttt{type} returns the type of activity (equlibriumGame).
    \item
      \texttt{results} returns the original submissions (with market
      price and points added).
    \item
      \texttt{schedules} returns a list containing the supply and demand
      schedules for each round.
    \item
      \texttt{equilibria} returns a list containing the equilibria for
      each round.
    \item
      \texttt{grades} returns the aggregated points ``won'' by each
      student for the entire activity.
    \end{itemize}
  \end{itemize}
\item
  Entry and exit game: \texttt{entryGame(sheet\ =\ NULL,\ ...)}

  \begin{itemize}
  \item
    Arguments:

    \begin{itemize}
    \tightlist
    \item
      \texttt{sheet} (required) is a character string url corresponding
      to the Google Sheets location containing the individual
      submissions.
    \end{itemize}
  \item
    Outputs:

    \begin{itemize}
    \tightlist
    \item
      \texttt{type} returns the type of activity (equlibriumGame).
    \item
      \texttt{results} returns the original submissions (with market
      price and points added).
    \item
      \texttt{rounds} returns the number of rounds in ``results.''
    \item
      \texttt{equilibria} returns a list containing the equilibria for
      each round.
    \item
      \texttt{grades} returns the aggregated points ``won'' by each
      student for the entire activity.
    \end{itemize}
  \end{itemize}
\item
  Bertrand game:
  \texttt{bertrandGame(sheet\ =\ NULL,\ a\ =\ 10,\ b\ =\ -1,\ c\ =\ 6,\ f\ =\ 0,\ ...)}

  \begin{itemize}
  \item
    Arguments:

    \begin{itemize}
    \tightlist
    \item
      \texttt{sheet} (required) is a character string url corresponding
      to the Google Sheets location containing the individual
      submissions.
    \item
      \texttt{a} is the value of the intercept of the linear
      inverse-demand function (default is 10).
    \item
      \texttt{b} is the value of the slope of the linear inverse-demand
      function (default is -1).
    \item
      \texttt{c} is the value of the firm's marginal cost (default is
      6).
    \item
      \texttt{f} is the value of the firm's fixed cost (default is 0).
    \end{itemize}
  \item
    Outputs:

    \begin{itemize}
    \tightlist
    \item
      \texttt{type} returns the type of activity (equlibriumGame).
    \item
      \texttt{results} returns the original submissions (with market
      price and points added).
    \item
      \texttt{grades} returns the aggregated points ``won'' by each
      student for the entire activity.
    \end{itemize}
  \end{itemize}
\item
  Cournot game:
  \texttt{cournotGame(sheet\ =\ NULL,\ a\ =\ 10,\ b\ =\ -1,\ c\ =\ 6,\ f\ =\ 0,\ ...)}

  \begin{itemize}
  \item
    Arguments:

    \begin{itemize}
    \tightlist
    \item
      \texttt{sheet} (required) is a character string url corresponding
      to the Google Sheets location containing the individual
      submissions.
    \item
      \texttt{a} is the value of the intercept of the linear
      inverse-demand function (default is 10).
    \item
      \texttt{b} is the value of the slope of the linear inverse-demand
      function (default is -1).
    \item
      \texttt{c} is the value of the firm's marginal cost (default is
      6).
    \item
      \texttt{f} is the value of the firm's fixed cost (default is 0).
    \end{itemize}
  \item
    Outputs:

    \begin{itemize}
    \tightlist
    \item
      \texttt{type} returns the type of activity (equlibriumGame).
    \item
      \texttt{results} returns the original submissions (with market
      price and points added).
    \item
      \texttt{grades} returns the aggregated points ``won'' by each
      student for the entire activity.
    \end{itemize}
  \end{itemize}
\item
  Stackelberg game:
  \texttt{stackelbergGame(sheet\ =\ NULL,\ a\ =\ 10,\ b\ =\ -1,\ c\ =\ 6,\ f\ =\ 0,\ ...)}

  \begin{itemize}
  \item
    Arguments:

    \begin{itemize}
    \tightlist
    \item
      \texttt{sheet} (required) is a character string url corresponding
      to the Google Sheets location containing the individual
      submissions.
    \item
      \texttt{a} is the value of the intercept of the linear
      inverse-demand function (default is 10).
    \item
      \texttt{b} is the value of the slope of the linear inverse-demand
      function (default is -1).
    \item
      \texttt{c} is the value of the firm's marginal cost (default is
      6).
    \item
      \texttt{f} is the value of the firm's fixed cost (default is 0).
    \end{itemize}
  \item
    Outputs:

    \begin{itemize}
    \tightlist
    \item
      \texttt{type} returns the type of activity (equlibriumGame).
    \item
      \texttt{results} returns the original submissions (with market
      price and points added).
    \item
      \texttt{grades} returns the aggregated points ``won'' by each
      student for the entire activity.
    \end{itemize}
  \end{itemize}
\end{itemize}

Another feature of the package is the ability to plot the results. The
syntax to plot any of the games described in this paper is simply
\texttt{plot(econGame,\ ...)}, where the (sole) argument is the name of
an object assigned by one of the \texttt{econGame} functions. The plot
the function generates depends on the type of game:

\begin{itemize}
\tightlist
\item
  For
  \texttt{type\ =\ \textquotesingle{}equilibriumGame\textquotesingle{}},
  plot the supply and demand functions with the corresponding equlibrium
  point.
\item
  For \texttt{type\ =\ \textquotesingle{}entryGame\textquotesingle{}},
  plot the supply, demand, and per-unit cost lines to show profits and
  losses.
\item
  For
  \texttt{type\ =\ \textquotesingle{}bertrandGame\textquotesingle{}},
  plot a histogram of the price strategies.
\item
  For \texttt{type\ =\ \textquotesingle{}cournotGame\textquotesingle{}}
  or
  \texttt{type\ =\ \textquotesingle{}stackelbergGame\textquotesingle{}},
  plot a bar graph of the strategy outcomes.
\end{itemize}

\hypertarget{shiny-user-interface}{%
\subsection{Shiny User Interface}\label{shiny-user-interface}}

The functions for directly summarizing the results of the games may be
useful for tabulating the results for the purposes of awarding points to
the students who participated, but may not be the most
visually-appealing way to present the results in class. To improve the
user interface for a ``prettier'' presentation of the results, I have
built a Shiny Application UI for each of the games. In this interface,
the instructor can display the raw results, plots, or schedules of the
outcomes of the results by inputting the sheet ID (and other parameters)
in the input boxes, and by switching the display tabs of the results.

To run the Shiny App for a given package, I have written a function that
executes the app from the
\texttt{\textquotesingle{}\textasciitilde{}/inst/shiny-examples\textquotesingle{}}
folder of the package source. To execute the app, all the instructor
needs to do is type one of the following commands:
\texttt{\textquotesingle{}runEquilibriumGameApp()\textquotesingle{}},
\texttt{\textquotesingle{}runEntryGameApp()\textquotesingle{}},
\texttt{\textquotesingle{}runBertrandGameApp()\textquotesingle{}},
\texttt{\textquotesingle{}runCournotGameApp()\textquotesingle{}}, or
\texttt{\textquotesingle{}runStackelbergGameApp()\textquotesingle{}}.
Figure 3 shows the Shiny interface for \texttt{equilibriumGame}.

\begin{figure}
\centering
\includegraphics{Figure3.png}
\caption{Shiny User Interface}
\end{figure}

\hypertarget{discussion}{%
\section{Discussion}\label{discussion}}

This work provided examples of how to implement some simple market games
using R and Shiny. The objective of the functions developed for this
discussion was to help instructors adopt more creative and engaging
teaching methods in their principles classrooms by lowering both the
financial and time costs of tabulating the results and presenting
appealing summaries of the outcomes.

All of the examples can be implemented for the students using Google
Forms, which collects the results in Google Sheets. Instructors may then
tabulate the results using the \texttt{econGame} package via the HTML
Shiny App with a single-function command in R. All of these applications
and packages come at no cost to the student. The only potential cost
barrier is the device - which could be as little as a mobile device -
each student would need in order to input their responses.

Another useful feature is that I have platformed the package on GitHub,
which allows users to make pull requests to suggest edits and changes to
improve or add features and examples to the package. Plans are already
in place to develop more examples from existing classroom experiments,
and even design a few original games. If you use these resources, please
let me know and share them with your colleagues.

\hypertarget{references}{%
\section*{References}\label{references}}
\addcontentsline{toc}{section}{References}

\hypertarget{refs}{}
\begin{CSLReferences}{1}{0}
\leavevmode\vadjust pre{\hypertarget{ref-brauer_games_2001}{}}%
Brauer, Jurgen, and Greg Delemeester. 2001. {``Games Economists Play:
{A} Survey of Non-Computerized Classroom-Games for College Economics.''}
\emph{Journal of Economic Surveys} 15 (2): 221--36.

\leavevmode\vadjust pre{\hypertarget{ref-chamberlin_experimental_1948}{}}%
Chamberlin, Edward H. 1948. {``An Experimental Imperfect Market.''}
\emph{Journal of Political Economy} 56 (2): 95--108.

\leavevmode\vadjust pre{\hypertarget{ref-deyoung_market_1993}{}}%
DeYoung, Robert. 1993. {``Market Experiments: {The} Laboratory Versus
the Classroom.''} \emph{The Journal of Economic Education} 24 (4):
335--51.

\leavevmode\vadjust pre{\hypertarget{ref-dickie_classroom_2006}{}}%
Dickie, Mark. 2006. {``Do {Classroom} {Experiments} {Increase}
{Learning} in {Introductory} {Microeconomics}?''} \emph{The Journal of
Economic Education} 37 (3): 267--88.
\url{https://doi.org/10.3200/JECE.37.3.267-288}.

\leavevmode\vadjust pre{\hypertarget{ref-emerson_comparing_2004}{}}%
Emerson, Tisha LN, and Beck A. Taylor. 2004. {``Comparing Student
Achievement Across Experimental and Lecture-Oriented Sections of a
Principles of Microeconomics Course.''} \emph{Southern Economic
Journal}, 672--93.

\leavevmode\vadjust pre{\hypertarget{ref-garratt_free_2000}{}}%
Garratt, Rod. 2000. {``A Free Entry and Exit Experiment.''}
\emph{Journal of Economic Education} 31 (3): 237.
\url{https://www.proquest.com/docview/235237213/citation/8DF4FB2251DE443BPQ/1}.

\leavevmode\vadjust pre{\hypertarget{ref-gremmen_assessing_1997}{}}%
Gremmen, Hans, and Jan Potters. 1997. {``Assessing the Efficacy of
Gaming in Economic Education.''} \emph{The Journal of Economic
Education} 28 (4): 291--303.

\leavevmode\vadjust pre{\hypertarget{ref-holt_industrial_1993}{}}%
Holt, Charles A. 1993. {``Industrial {Organization}: {A} {Survey} of
{Laboratory} {Research}.''}

\leavevmode\vadjust pre{\hypertarget{ref-holt_classroom_1996}{}}%
---------. 1996. {``Classroom Games: {Trading} in a Pit Market.''}
\emph{Journal of Economic Perspectives} 10 (1): 193--203.

\leavevmode\vadjust pre{\hypertarget{ref-smith_experimental_1962}{}}%
Smith, Vernon L. 1962. {``An Experimental Study of Competitive Market
Behavior.''} \emph{Journal of Political Economy} 70 (2): 111--37.

\leavevmode\vadjust pre{\hypertarget{ref-stodder_experimental_1998}{}}%
Stodder, James. 1998. {``Experimental Moralities: {Ethics} in Classroom
Experiments.''} \emph{The Journal of Economic Education} 29 (2):
127--38.

\leavevmode\vadjust pre{\hypertarget{ref-watts_little_2008}{}}%
Watts, Michael, and William E. Becker. 2008. {``A {Little} {More} Than
{Chalk} and {Talk}: {Results} from a {Third} {National} {Survey} of
{Teaching} {Methods} in {Undergraduate} {Economics} {Courses}.''}
\emph{The Journal of Economic Education} 39 (3): 273--86.
\url{https://doi.org/10.3200/JECE.39.3.273-286}.

\end{CSLReferences}

\end{document}
